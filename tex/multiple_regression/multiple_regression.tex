
% Inbuilt themes in beamer
\documentclass[aspectratio=169]{beamer}
\usepackage{graphicx}
% Theme choice:
\usetheme{CambridgeUS}

% Title page details: 
\title{Econometrics Discussion Section 2} 
\author{John Green}
\date{Spring 2024}


\begin{document}

% Title page
\begin{frame}
    \titlepage 
\end{frame}

% Outline frame
\begin{frame}{Omitted Variable Bias}
    \begin{itemize}
        \item In economics, we often want to determine the impact of $X$ on $Y$
        \begin{itemize}
            \item But $X$ is not the only thing going on! Usually many variables are effecting $Y$
            \item If we don't include these other variables in our model, we will get biased estimates of the effect of $X$ on $Y$, a problem referred to as \textit{omitted variable bias}
        \end{itemize}
        \item Let's think about what this does to our estimate with a silly example
        \begin{itemize}
            \item We are an AC company, and we want to know what causes people to buy more AC units
            \item Our statistician tells us that the number of swimming pool accidents is a good predictor of AC units sold
            \item We run a regression of AC units sold on swimming pool accidents and find a positive relationship
            \item Should we conclude that swimming pool accidents cause people to buy more AC units?
        \end{itemize} 
    \end{itemize}
\end{frame}

\begin{frame}{Omitted Variable Bias}
    \begin{itemize}
        \item Should we conclude that swimming pool accidents cause people to buy more AC units?
        \item Of course not! The omitted variable here is temperature. When it's hot, people buy more AC units and more people go swimming
        \item Intuitively, what direction do we expect the bias in the effect of swimming pool accidents to be?
    \end{itemize}
\end{frame}

\begin{frame}{Omitted Variable Bias}
    \begin{itemize}
        \item Should we conclude that swimming pool accidents cause people to buy more AC units?
        \item Of course not! The omitted variable here is temperature. When it's hot, people buy more AC units and more people go swimming
        \item Intuitively, what direction do we expect the bias in the effect of swimming pool accidents to be?
        \item It should bias the estimate \textit{upwards}: we are attributing the effect of temperature to swimming pool accidents and thus making swimming pool accidents look more important than they are
    \end{itemize}
\end{frame}

\begin{frame}{Omitted Variable Bias}
    \begin{itemize}
        \item We have omitted variable bias when we have a variable $Z$ that is correlated with $X$ and is a determinant of $Y$
        \begin{itemize}
            \item Note: if $Z$ is a determinant of $Y$ but is not correlated with $X$, then it is not a problem!
            \item These conditions mean that our OLS assumption of $E(u|X) = 0$ is violated
        \end{itemize}
        \item Formula for OVB:
        $$
        \hat{\beta} \to_{p} \beta + \frac{\sigma_u}{\sigma_X}\rho_{Xu}
        $$
        \begin{itemize}
            \item The intuition from our example shows up in the correlation term
        \end{itemize}
    \end{itemize}
\end{frame}

\begin{frame}{Causal effects}
    \begin{itemize}
        \item We can use OLS to summarize a relationship without attaching any directionality
        \begin{itemize}
            \item In this case we need to be careful in our language: "a change in X is associated with a change in Y"
        \end{itemize}
        \item Usually economists want to be able to say something \textit{causal}
        \item Ideal is a randomized controlled trial (RCT): some group gets the treatment, some group doesn't, and we compare outcomes between the two
        \begin{itemize}
            \item Observational data usually differs from this in important ways
        \end{itemize}
    \end{itemize}
\end{frame}

\begin{frame}{Solution to OVB}
    \begin{itemize}
        \item An RCT eliminates OVB because we randomly assign the treatment, which will then not be correlated with any other variables!
        \begin{itemize}
            \item This is usually not possible in economics --- this is why economics is hard!
        \end{itemize}
        \item Cross-tabulation eg run the regression on a subsample of your population where there is no OVB problem (but other issues emerge)
        \item Try and include omitted variables (obviously) in a multivariable regression
    \end{itemize}
\end{frame}

\begin{frame}{Multivariate regression}
    \begin{itemize}
        \item Same logic as before:
        $$
        Y_i = \beta_0 + \beta_1X_{1i} + \beta_2X_{2i} + u_i    
        $$
        \item Estimators derived in same way as before (just using matrices)
        \item Before we looked at $R^2$ as a measure of fit, but now we have to be careful: adding in more variables on the RHS will always increase $R^2$
        \item Adjusted $R^2$ is a better measure of fit which includes a degrees-of-freedom correction to penalize for adding in more variables
        \item Add one more assumption our previous 3 from the single-variable case: no perfect multicolinearity
    \end{itemize}
\end{frame}

\begin{frame}{Perfect multicolinearity}
    \begin{itemize}
        \item This is when one of your RHS variables is a perfect linear combination of others
        \item Why is this a problem?
    \end{itemize}
\end{frame}

\begin{frame}{Perfect multicolinearity}
    \begin{itemize}
        \item This is when one of your RHS variables is a perfect linear combination of others
        \item Why is this a problem?
        \item Think about basic algebra:
        \begin{itemize}
            \item $y = a*x$
            \item What is $a$?
        \end{itemize}
    \end{itemize}
\end{frame}

\begin{frame}{Perfect multicolinearity}
    \begin{itemize}
        \item $y = a*x$
        \item What is $a$?
        \item $a = \frac{y}{x}$
        \item But now what if I give you:
        \begin{itemize}
            \item $y = b*x + c*x $
            \item What are $a$ and $b$?
        \end{itemize}
    \end{itemize}
\end{frame}

\begin{frame}{Perfect multicolinearity}
    \begin{itemize}
        \item But now what if I give you:
        \begin{itemize}
            \item $y = b*x + c*x $
            \item What are $b$ and $c$?
        \end{itemize}
        \item This question has no unique answer! Any combination of $b$ and $c$ such that $a = b + c$ will work (an infinite number)
        \item This is what's going on when their is perfect multicolinarity: our coefficient estimates $\beta$ are not identified/unique
    \end{itemize}
\end{frame}

\begin{frame}{Binary variables}
    \begin{itemize}
        \item For this reason we need to be careful when we have binary (0/1) variables in our regression (or any kind of categorical variable)
        \item We either need to leave one out (the omitted group) or we need to drop our intercept
        \item Either is fine, just changes interpretation of the coefficient estimates
    \end{itemize}
\end{frame}

% \begin{frame}
%     \centering
%     \includegraphics[width = .6\textwidth,keepaspectratio]{./figs/age_ed.png}
% \end{frame}


\end{document}