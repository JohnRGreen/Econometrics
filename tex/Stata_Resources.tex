\documentclass{article}

% \aboveusepackage{preamble}
\usepackage{setspace}
\usepackage{amssymb}
\usepackage[colorlinks=true,allcolors=blue]{hyperref}
\usepackage{graphicx}
\usepackage{natbib}
\bibpunct{(}{)}{;}{a}{}{,}
\usepackage{caption}
\DeclareCaptionLabelSeparator{colquad}{:\quad}
\captionsetup{labelsep=colquad}
\usepackage{hyperref}

\usepackage{geometry}

\geometry{
    left=1in,
    right=1in,
    top=1in,
    bottom=1in
}

\doublespacing

\begin{document}

\begin{center}
    \Large
    \textbf{Stata Resources}
    \normalsize

    \vspace{1mm}

    Econometrics

    \vspace{1mm}

    Spring 2024
\end{center}
\medskip

This course will use Stata to perform statistical analyses on the problem sets. In addition, an understanding of Stata will be important for doing well on the exams (though you will not be asked to literally write out code).

\vspace{2mm}

Stata is an extremely user-friendly software program to analyze data. Much of its functionality is available through point-and-click commands, so how much you use the command line to code is up to you. It is a great skill to have on your resume and I recommend you take the time to learn it well.

\vspace{2mm}

JHU students can access Stata for free through their \href{http://mycloud.jh.edu/}{mycloud account}. Full instructions for setting this up are available \href{https://studentaffairs.jhu.edu/computing/campus-resources/myjlab/}{here}. If you have trouble, please reach out first to the Helpdesk support following the instructions on that page. They will be best able to help.

\vspace{2mm}

There are a number of free guides and introductions to Stata available online. Stata provides a \href{https://www.stata.com/links/resources-for-learning-stata/}{list of helpful resources}; I in particular will recommend German Rodriguez' \href{https://grodri.github.io/stata/}{Stata tutorial}. Spending 30-45 minutes going through an introduction will make the problem sets exponentially easier and I highly recommend doing so! That time is more than worth it in terms of time that you will save on the homework.

\end{document}